\documentclass[letterpaper]{article}
\usepackage[titletoc,toc,title]{appendix}

\usepackage{fancyheadings,multicol}
\usepackage{amsmath,amssymb}
\usepackage{listings}

\setlength{\textheight}{\paperheight}
\addtolength{\textheight}{-2in}
\setlength{\topmargin}{-.5in}
\setlength{\headsep}{.5in}
\addtolength{\headsep}{-\headheight}
\setlength{\footskip}{.5in}
\setlength{\textwidth}{\paperwidth}
\addtolength{\textwidth}{-2in}
\setlength{\oddsidemargin}{0in}
\setlength{\evensidemargin}{0in}
\flushbottom

\title{Semantic Role Labeling with Assistance of Neural Network Dependency Parser}
\author{Shawn Peng}


\begin{document}

\maketitle
\pagenumbering{gobble}
%	\newpage
\pagenumbering{arabic}

\flushleft

\section{Introduction}
``Semantic role labeling, sometimes also called shallow semantic parsing, is a task in natural language processing consisting of the detection of the semantic arguments associated with the predicate or verb of a sentence and their classification into their specific roles." [wiki]
These days, there emerged new approaches on dependency parser, we want to see how it will help the semantic role labeling task.
find relations and find roles for words.



\section{Related Works}
I mainly will use one previous achievement, A Fast and Accurate Dependency Parser using Neural Networks \cite{manning-EtAl:2014:P14-5}.

\bigskip
Reading list.

\bigskip
The paper Automatic Labeling of Semantic Roles (by Daniel Gildea and Daniel Jurafsky)\cite{Gildea:2002:ALS:643092.643093}, introduced a probabilistic approach to address the semantic role labeling job. It works on the FrameNet dataset and get quite feasible accuracy.

\bigskip
Charles J. Fillmore, and Collin Baker developed the theory of frame semantics that the FrameNet is based on. []

\bigskip
Semantic Role Labeling for Open Information Extraction by Janara Christensen, Mausam, Stephen Soderland and Oren Etzioni []

\bigskip
The Importance of Syntactic Parsing and	Inference in Semantic Role Labeling \cite{Punyakanok}

\bigskip
Semantic Role Labeling with Neural Network Factors \cite{FitzGerald2015SemanticRL}

\bigskip
Sentence Simplification for Semantic Role Labeling \cite{Woodsend:2014:TRI:2750423.2750426}

\bigskip
Frame-semantic parsing \cite{Das:2014}

\bigskip
Tools,\\
A Python wrapper for Stanford CoreNLP by Smitha Milli that uses the new CoreNLP v3.6.0 server. []

\section{Background}
``The goal of the semantic role labeling task is to discover the predicate–argument structure
of each predicate in a given input sentence."\cite{} In this work, we only focus on the verb predicate.

\section{Method}
We will try to implement a semantic role labeling program, mainly something based on the dependency parser proposed in Manning's paper \cite{manning-EtAl:2014:P14-5}. We maybe will implement some method in previous papers. Another option is for implement a new neural network for this task.

\section{Experiment}
For dependency parser, they report they have this accuracy,\\
Development: 	(1700 sentences)	UAS	92.0	LAS	89.7\\
Test:			(2416 sentences)	UAS	91.7	LAS	89.5

We don't have their original data, but maybe we could try to test on the data used by the paper CoNLL-2005 Shared Task.




\section{Conclusion}



\bibliographystyle{unsrt}
\bibliography{proj}

\begin{appendices}

\lstset{basicstyle=\ttfamily,
showstringspaces=false,
commentstyle=\color{red},
keywordstyle=\color{blue}
}

\section{Whole Process}
Start Stanford Dependency Parser server with the following command, \\
\lstinline [language=bash]{java -mx4g -cp "*" edu.stanford.nlp.pipeline.StanfordCoreNLPServer}.

% Evaluate the Stanford Dependency Parser on ANC dataset.
Evaluate the Stanford Dependency Parser on CoNLL09 trial dataset.

Change output format to CoNLL-U format.
\newcommand{\echo}[0]{echo}
\begin{lstlisting}[breaklines=true,language=bash]
java -mx150m -cp "stanford-parser-full-2013-06-20/*:" edu.stanford.nlp.parser.lexparser.LexicalizedParser -outputFormat "penn" edu/stanford/nlp/models/lexparser/englishPCFG.ser.gz stanford-parser-full-2013-06-20/data/testsent.txt >testsent.tree

java -mx150m -cp "stanford-parser-full-2013-06-20/*:" edu.stanford.nlp.trees.EnglishGrammaticalStructure -treeFile testsent.tree -conllx


\end{lstlisting}

MASC Corpus Header

\section{CoNLL-U format of Stanford nndep}
“conll”: A tab-separated values (TSV) format. Output extension is .conll. This representation may give only a partial view of an Annotation and doesn’t correspond to any particular CoNLL format. Columns are: wordIndex, token, lemma, POS, NER, head, depRel. \\

\bigskip
“conllu”: CoNLL-U output format, another tab-separated values (TSV) format. Output extension is .conllu. This representation may give only a partial view of an Annotation.\\

http://stanfordnlp.github.io/CoreNLP/cmdline.html\\

\bigskip
http://nlp.stanford.edu/nlp/javadoc/javanlp/edu/stanford/nlp/pipeline/CoNLLUOutputter.html

conll-u format reference[] \\

conll:   http://ilk.uvt.nl/conll/\#dataformat\\
conllu:  http://universaldependencies.org/docs/format.html\\

\section{Test Dataset}
OntoNotes 5.0\\
Use CoNLL 2012 to convert data format\\
http://conll.cemantix.org/2012/data.html\\



\end{appendices}


\end{document}
